\documentclass[12pt,a4paper]{book}
 \usepackage[utf8]{inputenc}
\usepackage[spanish]{babel}
\usepackage{amsmath} 
\usepackage{amsfonts}
\usepackage{amssymb}
\usepackage{graphicx}
\usepackage[left=2cm,right=2.5cm,top=2.5cm,bottom=2.5cm]{geometry} \begin{document}
\thispagestyle{empty}

\newcommand{\HRule}{\rule{\linewidth}{0.5mm}}


 {\centering
  
 \begin{figure}[htb] \centering

\includegraphics[scale=1]{Imagenes/upt.jpg}
 \end{figure}


\large{\bf UNIVERSIDAD PRIVADA DE TACNA}\\ \vspace{0.5cm}
\large{\bf FACULTAD DE INGENIERIA }\\ \vspace{0.5cm}

\large{\bf Escuela Profesional de Ingeniería de Sistemas } \\ \vspace{1cm}

{\large {\bf Laboratorio 2 - Unidad I }}\\ \vspace{1cm}

{\large {\bf “QLIK Q” }}\\ \vspace{2cm} 

\large Curso: INTELIGENCIA DE NEGOCIOS\\ \vspace{1.5cm}
\large Docente: Ing. Patrick Cuadros Quiroga \\ \vspace{1.5cm}
\large Ticona Mamani, Alex Armando (2017057860) \\ \vspace{4cm}
\vspace{0.5cm} {\Large {\bf \textsc{Tacna - Perú} }}\\ {\Large {\bf \textsc{2019}}} \\}



\begin{center}



\section*{ \\ PRACTICA DE LABORATORIO 02: \\ QLIK Q} 
\end{center}

\section*{Objetivo :}
Comprender la organización la información de nuestros datos de tal manera que todos los que los vean puedan comprender sus implicaciones y cómo actuar sobre ellos con claridad\\

\section*{crear una app}

\begin{itemize}
\item Inicie Qlik Sense.
\item Cree una nueva aplicación.
\item Nombre la aplicación: " Mi análisis de ventas "
\item Crear y abrir la aplicación 
\end{itemize}


\begin{center}
\includegraphics[width=12cm]{Imagenes/img1.jpg} 
\end{center}

\section*{Cargar datos}

Utilice el mosaico Agregar datos de archivos y otras fuentes para agregar el Datos de Excel a la aplicación (desde el archivo ExerciseData.xlsx ).o Cargue todos los campos de la hoja de cálculo, haciendo clic en Agregar datos botón.

\begin{center}
\includegraphics[width=12cm]{Imagenes/img2.jpg}
\end{center}

\begin{center}
\includegraphics[width=12cm]{Imagenes/img3.jpg}
\end{center}

\section*{Revisar datos}

\begin{itemize}
\item  Utilice el menú de acceso rápido para navegar a la vista Administrador de datos , en para obtener más información sobre esta tabla.
\item Abra la vista del editor de tablas del administrador de datos haciendo clic en el icono a lo largo de la parte inferior de la vista. Haga clic en los campos indicados y vea la tarjeta de resumen para responder a la siguientes preguntas 
 \end{itemize}
 
 
 ¿Cuántos valores distintos de países están presentes en esta tabla? :
 21\\
¿Cuántos valores distintos hay para la categoría de producto?¿campo? :
 8\\
¿Cuántos valores distintos hay para el campo Producto ? :
74\\
¿Cuántos valores distintos hay para el campo Fuente ? :
2\\
¿Cuáles son esos dos valores? :Existente Cuentas y cuentas nuevas\\


\begin{center}
\includegraphics[width=12cm]{Imagenes/img4.jpg}
\end{center}

\begin{center}
\includegraphics[width=12cm]{Imagenes/img5.jpg}
\end{center}
\subsection*{Desarrolle una hoja con el 'Asesor de conocimientos'}


\begin{itemize}
\item   Utilice el submenú de navegación de acceso rápido en Analizar> Hoja para seleccionar Insights , haga clic en el botón para
\item  Ubique el mapa en la lista de gráficos propuestos y Agregar a hoja > Mi nueva hoja .
\item Utilice el panel de la izquierda para seleccionar (casillas marcadas) los campos: Producto y Precio unitario .
\item Busque el gráfico de barras titulado: suma (PrecioUnitario) por Producto y Agregar a la hoja > Mi nueva hoja .
\item Borre los campos ().
\item Escriba lo siguiente en el campo ' Háganos una pregunta ': ¿Qué país tiene la mayor cantidad? (y enviar pregunta).
\item Busque el gráfico de barras titulado: suma (Cantidad) por país y Agregar a la hoja > Mi nueva hoja .
\item Utilice el submenú de navegación de acceso rápido en Analizar> Perspectivas para seleccionar Hoja y evaluar hoja que ha creado. Debería
 aparecer como ve a continuación:

 \end{itemize}

\begin{center}
\includegraphics[width=12cm]{Imagenes/img6.jpg}
\end{center}

\begin{center}
\includegraphics[width=12cm]{Imagenes/img7.jpg}
\end{center}

\subsection*{Desarrolle una hoja usando 'Ayuda para sugerencias de gráficos' (continuación)}

\begin{itemize}
\item Expanda el campo OrderDate en el panel de activos para exponer agrupaciones de fechas derivadas
\item o	Arrastre y suelte la agrupación YearMonth en un espacio vacío en el área de visualización
\item Una tabla de resultados tipo gráfico.
\item Arrastre y suelte el campo Ventas para cubrir toda la tabla que actualmente muestra los valores de año y mes para pedidos.
\item  Resultados de un gráfico de líneas, debido a el hecho de que una cita dimensión y un campo de los valores de medida han sido suministrado a esta visualización.

\end{itemize}

\begin{center}
\includegraphics[width=12cm]{Imagenes/img8.jpg}
\end{center}

\subsection*{Pruebe un tipo de gráfico alternativo sugerido)}

\begin{itemize}
\item Vuelva al modo Editar hoja y considere sugerencias de gráficos alternativos
\begin{center}
\includegraphics[width=12cm]{Imagenes/img9.jpg}
\end{center}
\item Tenga en cuenta que el panel de propiedades de los gráficos configurados con ayuda presenta solo configuraciones limitadas
\item Nos gustaría ajustar el esquema de color aplicado a las formas de los países en el gráfico de mapa, así que cambie el Las sugerencias de gráficos cambian a "desactivado".
\item Abra la sección Capas del panel de propiedades y haga clic en la capa Área del país .
\item Abra la sección Colores y localice las opciones de Esquema de colores .
\item Cambiar la combinación de colores del gradiente secuencial de divergencia de gradiente 
\item Marque la casilla de verificación Invertir colores 

\begin{center}
\includegraphics[width=12cm]{Imagenes/img10.jpg}
\end{center}
\end{itemize}

\subsection*{Desarrollar una hoja sin ayuda)}

\begin{itemize}
\item Cree una nueva hoja en la aplicación, titulada " Mis gráficos personalizados ".
\item Arrastre y suelte objetos desde la sección Gráficos () del panel de activos al área de visualización, y cambiar el tamaño de cada uno, para rellenar una hoja con un Gráfico de barras , tabla , gráfico de sectores , KPI , panel Filtro , y texto y área de la imagen organizada dentro del espacio de visualización como se muestra a continuación:

\end{itemize}

\begin{center}
\includegraphics[width=12cm]{Imagenes/img11.jpg}
\end{center}

\begin{center}
\includegraphics[width=12cm]{Imagenes/img12.jpg}
\end{center}

\subsection*{Analizar una hoja)}

\begin{itemize}
\item Seleccione el año 2013 , tome una instantánea del gráfico circular, anote con " 2013 ".
\item Seleccione el año 2015 y tome una instantánea del gráfico circular, anote con " 2015 ".

\item Seleccione la barra Traje de baño en el gráfico de barras y confirme la selección.
\item Seleccione la sección Cuentas nuevas en el gráfico circular y confirme la selección.
\item luego crea un nuevo marcador:

\item Nombre el marcador = " pedidos de 2014 "
\item Utilice los dos marcadores para alternar entre estos diferentes estados de los datos seleccionados

\end{itemize}

\begin{center}
\includegraphics[width=12cm]{Imagenes/img13.jpg}
\end{center}

\begin{center}
\includegraphics[width=12cm]{Imagenes/img14.jpg}
\end{center}

\begin{center}
\includegraphics[width=12cm]{Imagenes/img15.jpg}
\end{center}

\begin{center}
\includegraphics[width=12cm]{Imagenes/img16.jpg}
\end{center}

\begin{center}
\includegraphics[width=12cm]{Imagenes/img17.jpg}
\end{center}


\end{document}


